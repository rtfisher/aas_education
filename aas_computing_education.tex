\documentclass[11pt]{article}

\usepackage{mathpazo}

\usepackage[margin=1in]{geometry}

\usepackage{hyperref}

\usepackage[defaultsans]{cantarell}
\usepackage[T1]{fontenc}


\usepackage[small,compact]{titlesec}
\titlespacing{\subsection}{0pt}{*2}{*0.125}
\titleformat*{\subsection}{\sffamily\bfseries}
\titleformat*{\section}{\sffamily\bfseries}

\usepackage{fancyhdr}
\pagestyle{fancy}
\rhead{\footnotesize \sf The Importance of Computation in Astronomy Education}
\lhead{}

\usepackage{tcolorbox}

\begin{document}

\thispagestyle{plain}

\begin{center}
{\Large \sffamily \bfseries The Importance of Computation in Astronomy Education} \\
{M. Zingale\footnote{Stony Brook University}}
\end{center}

\begin{tcolorbox}
{\sffamily \bfseries Executive Summary:} Computational skills are required
across all disciplines, theory and observation inclusive, in astronomy.
Many students enter graduate programs without sufficient skills
to solve computational problems in their core classes and to jump in and
contribute right away in research.  We recommend a push for computational
literacy in the early undergraduate years, and familiarity with fundamental 
software carpentry skills as well as core numerical methods by the
completion of an undergraduate degree in Astronomy. Further, we recommend
that these skills be built on throughout graduate education by including
computational problems in the core astronomy classes.  Finally, we discuss
the role of open source software in astronomy education.
\end{tcolorbox}

\section{Computational Needs in Astronomy \& Astrophysics}

Computational skills are required at all levels of research in
astronomy.  Theory is dominated by simulation codes, e.g. stellar
evolution or multidimensional, multiphysics hydrodynamics codes,
written in a variety of languages (C++, Fortran, and Python being the
most popular).  Observational astronomy is entirely digital, and the
workflow takes the form of a software pipeline to reduce and analyze
data.  IDL was a popular player here a decade ago, but is rapidly
being replaced by Python, and core libraries such as AstroPy are being
developed to fill this need.

Furthermore, the workflow in Astronomy is often expressed in terms of
a UNIX-like environment with OS X or Linux serving as the OS of
choice.  Students coming out of high school may be unfamiliar with
(and put off from) the commandline and the power it enables.

\section{Undergraduate Education}

SUNY transfer path for physics requires intro to CS in first 2 years

 SBU majors require intro class

Each year many students are introduced to astronomy \& astrophysics through
introductory-level courses designed both for astronomy \& astrophysics majors
as well as a larger pool of physics majors. Through this core course, students
are introduced for the first time to the most fundamental core topics in astronomy 
\& astrophysics, ranging from our own solar system to exoplanetary systems, 
stellar structure and evolution, galactic dynamics, large-scale structure, and cosmology.

Open data archives enable access to galactic and extragalactic data relevant to all of these
core topics, and present educators with outstanding opportunities to bring students directly
into contact with real-world data, and to integrate data analysis  and computation into the
standard undergraduate curriculum. Some examples of data-driven exercises include :


\begin {enumerate} 

\item Inferring the mass, radius, and density of the historic transiting exoplanet HD209458b

\item Creation of a HR diagram from Tycho data

\item Calculation of stellar interiors using MESA-web

\item Determination of the value of the Hubble constant $H_0$ from Type Ia light curve data

\item Analysis of the gravitational waves from the historic binary black hole merger GW150914

\end {enumerate}

Instructors can guide students with highly-structured homework exercises and class projects which 
empower students to become acquainted with the power of computation on realistic ``real world''
examples, and learn foundational data analysis skills.  

Broad dissemination of software carpentry lesson training plans, exemplified
by recent AAS workshops on this topic, will help familiarize students with 
these fundamental skills. 

\section{Graduate Education}

At the graduate level, a popular way to train students in the specialized
codes and techniques used in each subdiscipline are summer schools.

A common issue at the graduate level is that speciality classes (e.g.,
one focusing on computational hydrodynamics) tend to attract only a
small number of students, making it difficult to justify their regular
offering.

\subsection{High-Performance Computing}

Summer schools


\section{Open Source and Education}

\subsection{Python and Jupyter Notebooks}

Notebooks have found wide adoption in the classroom and allow for
interactive in-class activities and for the student to reply the
lecture on their own outside of class.

\subsection{Simulation Codes}




\section{Careers}

Many Astronomy PhDs will not stay in academia, since each faculty member
will produce many PhDs during their academic career.  Computational skills are perhaps
among the most transferable skills which an astronomy graduate student will acquire
in their education and training, and
can make a grad very attractive to a wide range of potential employers.



\end{document}
